\include{wp-preamble}
% ======================================================================
\begin{document}
% ======================================================================
\title{Recursive structure of primes reloaded. Now with mod 2, too.\footnote{\keywords{integer divisible numbers, prime numbers, recursive structure, mod 2}; \license{CC BY-ND 3.0 DE (see also LICENSE), No warranty for mistakes!}}}
\date{\today}

\author{Carolin Z\"obelein\footnote{\authoremail{contact@carolin-zoebelein.de}{D4A7 35E8 D47F 801F 2CF6 2BA7 927A FD3C DE47 E13B}; \authorurl{http://www.carolin-zoebelein.de}}}
% ======================================================================
\maketitle
\begin{abstract}
	Abstract.
\end{abstract}
\tableofcontents
%\newpage
% ======================================================================
% Introduction
\section{Introduction}
\label{s:introduction}
% ----------------------------------------------------------------------
In this working document I will make a revision of my work \cite{2014arXiv1411.2824Z}. Now for mod 2 and not mod 6 which makes all so easy. Sometimes I'm so stupid. wtf!
% ----------------------------------------------------------------------
% Road map
\section{Road map}
\label{s:roadmap}
% ----------------------------------------------------------------------
Since the main steps will be the same like in \cite{2014arXiv1411.2824Z} I will only make a few comments. If you need more information please read the arXiv paper at first. \\

To future comments: Yes, I know it's no 'professional' math, but why I should use abstract math if it's also possible to explain issues with easy tools. This here is to be only a first impression. Not the solution of all! After this you can construct your buildings, too! \\ 
  
Personal road map for this document:
\begin{enumerate}
	\item Formulation of the important points for mod 2.
	\item Consequences from this.
	\item $\dots$
\label{en:roadmap}\end{enumerate}
% ----------------------------------------------------------------------
% What we already know ...
\section{What we already know ...}
\label{s:whatwealreadyknow}
% ----------------------------------------------------------------------
Short repetition of the properties of primes which we already know and we will use for our work.
\begin{enumerate}
	\item Q: What are prime numbers $\mathbb{P}$? \\
		A: A prime number $p \in \mathbb{P}$ is a integer number lager than one which has no positive integer divisors apart from $1$ and itself. \\
		$\Rightarrow$ More mathematical: $p \in \mathbb{Z}:\mathrm{gcd}\left(n,p\right) = 1, \forall n \neq p \in \mathbb{N}$
	\item The set of primes: $\mathbb{P} := \{ 2, 3, 5, 7, 11, 13, \dots\}$
	\item The number $2$ is the only even prime.
\label{en:propertiesprimes}\end{enumerate}

\textbf{In the following work we will always ignore the prime 2!}









% ======================================================================
% Bibliography
\nocite{*}
\newpage
%\bibliographystyle{amsplain}
\bibliographystyle{unsrtdin}
\bibliography{wp-primesrec-mod2}
% ======================================================================
\end{document}
