\include{wp-preamble}
% ======================================================================
\begin{document}
% ======================================================================
\title{Recursive structure of primes reloaded. Now with mod 2, too.\footnote{\keywords{integer divisible numbers, prime numbers, recursive structure, mod 2}; \license{CC BY-ND 3.0 DE (see also LICENSE), No warranty for mistakes!}}}
\date{\today}

\author{Carolin Z\"obelein\footnote{\authoremail{contact@carolin-zoebelein.de}{D4A7 35E8 D47F 801F 2CF6 2BA7 927A FD3C DE47 E13B}; \authorurl{http://www.carolin-zoebelein.de}}}
% ======================================================================
\maketitle
\begin{abstract}
	Abstract.
\end{abstract}
\tableofcontents
%\newpage
% ======================================================================
% Introduction
\section{Introduction}
\label{s:introduction}
% ----------------------------------------------------------------------
In this working document I will make a revision of my work \cite{2014arXiv1411.2824Z}. Now for mod 2 and not mod 6 which makes all so easy. Sometimes I'm so stupid. wtf!
% ----------------------------------------------------------------------
% Road map
\section{Road map}
\label{s:roadmap}
% ----------------------------------------------------------------------
Since the main steps will be the same like in \cite{2014arXiv1411.2824Z} I will only make a few comments. If you need more information please read the arXiv paper at first. \\

To future comments: Yes, I know it's no 'professional' math, but why I should use abstract math if it's also possible to explain issues with easy tools. This here is to be only a first impression. Not the solution of all! After this you can construct your buildings, too! \\ 
  
Personal road map for this document:
\begin{enumerate}
	\item Formulation of the important points for mod 2.
	\item Consequences from this.
	\item $\dots$
\label{en:roadmap}\end{enumerate}
% ----------------------------------------------------------------------
% What we already know ...
\section{What we already know ...}
\label{s:whatwealreadyknow}
% ----------------------------------------------------------------------
Short repetition of the properties of primes which we already know and we will use for our work.
\begin{enumerate}
	\item Q: What are prime numbers $\mathbb{P}$? \\
		A: A prime number $p \in \mathbb{P}$ is a integer number lager than one which has no positive integer divisors apart from $1$ and itself. \\
		$\Rightarrow$ More mathematical: $p \in \mathbb{Z}:\mathrm{gcd}\left(n,p\right) = 1, \forall n \neq p \in \mathbb{N}$
	\item The set of primes: $\mathbb{P} := \{ 2, 3, 5, 7, 11, 13, \dots\}$
	\item The number $2$ is the only even prime.
\label{en:propertiesprimes}\end{enumerate}

\textbf{In the following work we will always ignore the prime 2!} \\

$\Rightarrow$ Hence, for all primes $p \in \mathbb{P} \setminus \{2\}$ we know that we are able to write them as $p_{n} := 2n + 1$, $n \in \mathbb{N}_{0}$. Q: For which $n$ we receive primes? \\

Now if we look at the set of all odd integer divisible numbers $p_{nn\prime} \in \mathbb{N}$. We recognise
\begin{equation}\begin{split}
	p_{nn\prime} & := \left(2n + 1\right)\left(2n\prime + 1\right) \\
	& = 4nn\prime + 2n + 2n\prime + 1 \\
	& = 2\left(\underbrace{2nn\prime + n + n\prime}_{=:N_{n,n\prime}}\right) + 1,
\end{split}\label{eq:intdivnumbN}\end{equation} 
$\forall n,n\prime \in \mathbb{N}_{0}$. Since the case $a \cdot 1 = 1 \cdot a = a$ isn't interesting for use we will change our domain of definition from $n,n\prime \in \mathbb{N}_{0}$ to $n,n\prime \in \mathbb{N}$. \\
Additionally we will see that it make sense to expand $p_{n}$ to $p_{z}$ with the set $\mathbb{Z}$. In this case we will write
\begin{equation}\begin{split}
	p_{z,z\prime} & := \left(2z + 1\right)\left(2z\prime + 1\right) \\
	& = 4zz\prime + 2z + 2z\prime + 1 \\
	& = 2\left(\underbrace{2zz\prime + z + z\prime}_{=:Z_{z,z\prime}}\right) + 1,
\end{split}\label{eq:intdivnumbZ}\end{equation}
$\forall z,z\prime \in \mathbb{Z}$. Here we are also not interested in the cases $a \cdot 1 = 1 \cdot a = a$ and $a \cdot \left(-1\right) = \left(-1\right) \cdot a = -a$. So we use $z,z\prime \in \mathbb{Z} \setminus \{-1, 0\}$.
% ----------------------------------------------------------------------
% Intersection
\section{Intersection}
\label{s:intersection}
% ----------------------------------------------------------------------
Ok. Now we have an equation to describe all integer divisible numbers of $\mathbb{Z}$. So we also write
\begin{equation}\begin{split}
	Z_{z,z\prime} & = 2zz\prime + z + z\prime \\
	& = \left(2z + 1\right)z\prime + z \\
	& = \left(2z\prime + 1\right)z + z\prime.	
\end{split}\label{eq:intdivnumbZ2}\end{equation}
For example, the second line can be interpreted as the equation which gives us all $Z_{z,z\prime}$ which belongs to numbers which are the product of $\left(2z + 1\right)V$, $V \in \mathbb{Z}$. \\
We use this to write the following for all numbers which are not integer divisible by $\left(2z + 1\right)$:
\begin{equation}
	Z_{z,z\prime} = \left(2z + 1\right)z\prime + z + \chi_{z},
\label{eq:notintdivZ}\end{equation}
$\chi_{z} \in [1, \left(2z + 1\right) - 1]$. For the intersection of two of this equations follows
\begin{equation}\begin{split}
	0 & = Z_{z_{i},z_{i}\prime} - Z_{z_{j},z_{j}\prime} \\
	& = \left(2z_{i} + 1\right)z_{i}\prime - \left(2z_{j} + 1\right)z_{j}\prime + \underbrace{z_{i} + \chi_{z_{i}}}_{=:\kappa_{z_{i}}} - \left(\underbrace{z_{j} + \chi_{z_{j}}}_{=:\kappa_{z_{j}}}\right) \\
	& = \left(2z_{i} + 1\right)z_{i}\prime - \left(2z_{j} + 1\right)z_{j}\prime + \kappa_{z_{i}} - \kappa_{z_{j}},
\end{split}\label{eq:intersection}\end{equation}
$\forall i,j \in \mathbb{N}: i \neq j$.
% ----------------------------------------------------------------------
% Solution of intersection equation
\section{Solution of intersection equation}
\label{s:solutionofintersection}
% ----------------------------------------------------------------------
Assume $\vert\left(2z_{i} + 1\right)\vert < \vert\left(2z_{j} + 1\right)\vert$ and $\left(2z_{i} + 1\right) \perp \left(2z_{j} + 1\right)$. Be $z_{j} = z_{i} + \Delta z_{i,j}:\Delta z_{i,j} \in \mathbb{N}$ we can write
\begin{equation}\begin{split}
	0 & = \left(2z_{i} + 1\right)z_{i}\prime - \left(2z_{j} + 1\right)z_{j}\prime + \kappa_{\left(z_{i}, \chi_{z_{i}}\right)} - \kappa_{\left(z_{j}, \chi_{z_{j}}\right)} \\
	& = \left(2z_{i} + 1\right)\left(z_{i}\prime - z_{j}\prime\right) - 2\Delta z_{j} + \chi_{z_{i}} - \chi_{z_{j}},
\end{split}\label{eq:intersectionDeltaz}\end{equation}
with $\chi_{z_{i}} \in [ 1, 2z_{i}]$ and $\chi_{z_{j}} \in [1, 2\left(z_{i} + \Delta z_{i,j}\right)]$. Like in \cite{2014arXiv1411.2824Z} we receive for $z_{j}\prime$
\begin{equation}
	z_{j}\prime^{\Delta z_{i,j}} = \left(2z_{i} + 1\right)Y_{j \circ i} - \left(\chi_{z_{i}} - \chi_{z_{j}}\right)z_{i}\prod_{k=2}^{\Delta z_{i,j}} \left(1 + \frac{2}{k}z_{i}\right)
\label{eq:zjsolution}\end{equation}
and in the same way for $z_{i} = z_{j} - \Delta z_{i,j}$ we receive for $z_{i}^{\prime}$
\begin{equation}
	z_{i}\prime^{\Delta z_{i,j}} = \left(2z_{j} + 1\right)Y_{j \circ i} - \left(\chi_{z_{i}} - \chi_{z_{j}}\right)z_{j}\prod_{k=2}^{\Delta z_{i,j}} \left(1 + \frac{2}{k}z_{j}\right).
\label{eq:zjsolution}\end{equation}
Finally we can write
\begin{equation}
	\prod_{k=2}^{\Delta z_{i,j}} \left(1 + \frac{2}{k}z_{i,j}\right) = \frac{\Gamma\left(2z_{i,j} + \Delta z_{i,j} + 1\right)}{\Gamma\left(\Delta z_{i,j} + 1\right)\Gamma\left(2z_{i,j} + 2\right)}
\label{eq:gammafunczsol}\end{equation}
% ----------------------------------------------------------------------
% ...
\section{...}
\label{s:...}
% ----------------------------------------------------------------------
Until now all was a repetition of \cite{2014arXiv1411.2824Z}. Question: How does the way with $2$ make all easier now?





% ======================================================================
% Bibliography
\nocite{*}
\newpage
%\bibliographystyle{amsplain}
\bibliographystyle{unsrtdin}
\bibliography{wp-primesrec-mod2}
% ======================================================================
\end{document}
